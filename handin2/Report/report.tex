%%%%%%%%%%%%%%%%%%%%%%%%%%%%%%%%%%%%%%%%%
% Short Sectioned Assignment
% LaTeX Template
% Version 1.0 (5/5/12)
%
% This template has been downloaded from:
% http://www.LaTeXTemplates.com
%
% Original author:
% Frits Wenneker (http://www.howtotex.com)
%
% License:
% CC BY-NC-SA 3.0 (http://creativecommons.org/licenses/by-nc-sa/3.0/)
%
%%%%%%%%%%%%%%%%%%%%%%%%%%%%%%%%%%%%%%%%%

%----------------------------------------------------------------------------------------
%	PACKAGES AND OTHER DOCUMENT CONFIGURATIONS
%----------------------------------------------------------------------------------------

\documentclass[paper=a4, fontsize=11pt]{scrartcl} % A4 paper and 11pt font size

\usepackage[T1]{fontenc} % Use 8-bit encoding that has 256 glyphs
\usepackage{fourier} % Use the Adobe Utopia font for the document - comment this line to return to the LaTeX default
\usepackage[english]{babel} % English language/hyphenation
\usepackage{amsmath,amsfonts,amsthm} % Math packages
\usepackage{csvsimple}
\usepackage{graphicx}

\usepackage{sectsty} % Allows customizing section commands
\allsectionsfont{\centering \normalfont\scshape} % Make all sections centered, the default font and small caps

\usepackage{fancyhdr} % Custom headers and footers
\pagestyle{fancyplain} % Makes all pages in the document conform to the custom headers and footers
\fancyhead{} % No page header - if you want one, create it in the same way as the footers below
\fancyfoot[L]{} % Empty left footer
\fancyfoot[C]{} % Empty center footer
\fancyfoot[R]{\thepage} % Page numbering for right footer
\renewcommand{\headrulewidth}{0pt} % Remove header underlines
\renewcommand{\footrulewidth}{0pt} % Remove footer underlines
\setlength{\headheight}{13.6pt} % Customize the height of the header

\numberwithin{equation}{section} % Number equations within sections (i.e. 1.1, 1.2, 2.1, 2.2 instead of 1, 2, 3, 4)
\numberwithin{figure}{section} % Number figures within sections (i.e. 1.1, 1.2, 2.1, 2.2 instead of 1, 2, 3, 4)
\numberwithin{table}{section} % Number tables within sections (i.e. 1.1, 1.2, 2.1, 2.2 instead of 1, 2, 3, 4)

\setlength\parindent{0pt} % Removes all indentation from paragraphs - comment this line for an assignment with lots of text

%----------------------------------------------------------------------------------------
%	TITLE SECTION
%----------------------------------------------------------------------------------------

\newcommand{\horrule}[1]{\rule{\linewidth}{#1}} % Create horizontal rule command with 1 argument of height

\title{	
\normalfont \normalsize 
%\textsc{university, school or department name} \\ [25pt] % Your university, school and/or department name(s)
\horrule{0.5pt} \\[0.4cm] % Thin top horizontal rule
\huge Handin 1 \\ % The assignment title
\horrule{2pt} \\[0.5cm] % Thick bottom horizontal rule
}

\author{S\o ren Meldgaard 201303712, Malthe Bisbo 201303718} % Your name

\date{\normalsize\today} % Today's date or a custom date

\begin{document}

\maketitle % Print the title

\section{PART I: Support Vector Machines}
\csvautotabular{svc_lin.csv} \\ \\
\csvautotabular{svc_poly2.csv} \\ \\
\csvautotabular{svc_poly3.csv} \\ \\
\csvautotabular{svc_rbf.csv} \\ \\
\subsection{Statistics}

\section{PART II: Neural Networks}
Below is the confusion matrix for the neural network. \\ \\
\csvautotabular{nn_256_confusion_matrix.csv} \\ \\
Below is the accuracy \\
train accuracy test accuracy \\                                  
0.9854527938342967   0.958139534883721  \\

\subsection{Statistics}
\begin{figure}[h]
\center
\includegraphics[trim={0 0cm 0 0cm},clip]{../results/nn_256_early_stopping.png}
\end{figure}

\subsection{Theory}

Parameters in a neural network: \\
U: 256(h) x 10(number of output nodes) \\
W: 786(input dimension) x 256(h) \\
b1: 256(h) \\
b2: 10 (number of output nodes) \\
Total parameters are the sum of the above. \\ \\

Operations in a forwad pass: \\
Each parameter is used exactly once in a forward pass. Assuming we have N inputs (images) then we simply use N times the above number of operations.

\section{PART III: Convolutional Neural Networks}

\subsection{Statistics}
Below is the confusion matrix for the neural network. \\ \\
\csvautotabular{cnn_1024_confusion_matrix.csv} \\ \\
Below is the accuracy \\
train accuracy test accuracy\\
0.9972061657032756, 0.9798449612403101

\subsection{Statistics}
\begin{figure}[h]
\center
\includegraphics[trim={0 0cm 0 0cm},clip]{../results/cnn_1024_early_stopping.png}
\end{figure}
\subsection{Theory}
Parameters in a convolutional neural network: \\
First convolution: 5 x 5 x 32 (assuming same weights all over the image) \\
Second convolution: 5 x 5 x 32 x 64 \\
W-matrix: 7 x 7 x 64 x 1064(hidden layer size)?? \\
U-matrix: 1064 x 10 \\
b1: 32\\
b2: 64\\
b3: 1064\\
b4: 10\\
\\ \\
Operations in convolutional neural networks: \\
First convolution: 5 x 5 x 28 x 28 x 32 \\
Adding b1: 32 x 28 x 28\\
First relu: 32 x 28 x 28 \\
Pool 1: 4 x 14 x 14 x 32 (assuming 4 operations for each small pool) \\
Second convolution: 5 x 5 x 14 x 14 x 32 x 64 \\
Adding b2: 64 x 14 x 14 \\
Second relu: 64 x 14 x 14 \\
Pool 2: 4 x 7 x 7 x 64 \\
Matrix mul:  7 x 7 x 64 x 1064 \\
Adding b3: 1064 \\
Third relu: 1064 \\
Matrix mul: 1064 x 10 \\
Adding b4: 10 \\
This was for one input. 




% Example of figure
%\begin{figure}[h]
%\center
%\includegraphics[trim={0 4.5cm 0 5cm},clip]{logistic_regression_parameter_plot_1_128.png}
%\end{figure}

% Example of csv table
% \csvautotabular{softmax_confusion_matrix.csv} \\ \\ \\


\end{document}