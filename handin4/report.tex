%%%%%%%%%%%%%%%%%%%%%%%%%%%%%%%%%%%%%%%%%
% Short Sectioned Assignment
% LaTeX Template
% Version 1.0 (5/5/12)
%
% This template has been downloaded from:
% http://www.LaTeXTemplates.com
%
% Original author:
% Frits Wenneker (http://www.howtotex.com)
%
% License:
% CC BY-NC-SA 3.0 (http://creativecommons.org/licenses/by-nc-sa/3.0/)
%
%%%%%%%%%%%%%%%%%%%%%%%%%%%%%%%%%%%%%%%%%

%----------------------------------------------------------------------------------------
%	PACKAGES AND OTHER DOCUMENT CONFIGURATIONS
%----------------------------------------------------------------------------------------

\documentclass[paper=a4, fontsize=11pt]{scrartcl} % A4 paper and 11pt font size

\usepackage[T1]{fontenc} % Use 8-bit encoding that has 256 glyphs
\usepackage{fourier} % Use the Adobe Utopia font for the document - comment this line to return to the LaTeX default
\usepackage[english]{babel} % English language/hyphenation
\usepackage{amsmath,amsfonts,amsthm} % Math packages
\usepackage{csvsimple}
\usepackage{graphicx}
\usepackage{booktabs}

\usepackage{sectsty} % Allows customizing section commands
\allsectionsfont{\centering \normalfont\scshape} % Make all sections centered, the default font and small caps

\usepackage{fancyhdr} % Custom headers and footers
\pagestyle{fancyplain} % Makes all pages in the document conform to the custom headers and footers
\fancyhead{} % No page header - if you want one, create it in the same way as the footers below
\fancyfoot[L]{} % Empty left footer
\fancyfoot[C]{} % Empty center footer
\fancyfoot[R]{\thepage} % Page numbering for right footer
\renewcommand{\headrulewidth}{0pt} % Remove header underlines
\renewcommand{\footrulewidth}{0pt} % Remove footer underlines
\setlength{\headheight}{13.6pt} % Customize the height of the header

\numberwithin{equation}{section} % Number equations within sections (i.e. 1.1, 1.2, 2.1, 2.2 instead of 1, 2, 3, 4)
\numberwithin{figure}{section} % Number figures within sections (i.e. 1.1, 1.2, 2.1, 2.2 instead of 1, 2, 3, 4)
\numberwithin{table}{section} % Number tables within sections (i.e. 1.1, 1.2, 2.1, 2.2 instead of 1, 2, 3, 4)

\setlength\parindent{0pt} % Removes all indentation from paragraphs - comment this line for an assignment with lots of text

%----------------------------------------------------------------------------------------
%	TITLE SECTION
%----------------------------------------------------------------------------------------

\newcommand{\horrule}[1]{\rule{\linewidth}{#1}} % Create horizontal rule command with 1 argument of height

\title{	
\normalfont \normalsize 
%\textsc{university, school or department name} \\ [25pt] % Your university, school and/or department name(s)
\horrule{0.5pt} \\[0.4cm] % Thin top horizontal rule
\huge Handin 1 \\ % The assignment title
\horrule{2pt} \\[0.5cm] % Thick bottom horizontal rule
}

\author{S\o ren Meldgaard 201303712, Malthe Bisbo 201303718} % Your name

\date{\normalsize\today} % Today's date or a custom date

\begin{document}

\maketitle % Print the title
\section*{Status of work}


\section*{Section 1}
The two algorithms was compared both by visual inspection and using the two scores. The visual inspection was carried out by inspecting 2d scatter plot with the data coloured according to the cluster it was assigned to. From visual inspection Lloyd's algorithm seems most promising, as it separated the data into well defined groups, while both the basic EM and the Lloyd initialized EM had a tendency to return overlapping distributions, which resulted in split clusters, when performing the hard clustering (assigning each point to the cluster with the largest probability).


\section*{Section 2.1 - Silhouette coefficients}
The silhouette coefficients does not wary much with either k or algorithm. For Lloyd's it scores highest for k=4 and EM scores highest for k=2.

\begin{table}[]
	\centering
	\caption{My caption}
	\label{my-label}
	\begin{tabular}{@{}llll@{}}
		\toprule
		& k=2    & k=3    & k=4    \\ \midrule
		Lloyd's & 0.9947 & 0.9966 & 0.9970 \\
		EM      & 0.9955 & 0.9926 & 0.9947
	\end{tabular}
\end{table}

\section*{Section 2.2 - F1 score}
Lloyd's algorithm scores the most for k=3 and EM for k=2. As the data contained three different labels it seems preferable that the algorithms also has the highest F1 score for k=3, which is only true for Lloyd's.

\begin{table}[]
	\centering
	\caption{My caption}
	\label{my-label}
	\begin{tabular}{@{}llll@{}}
		\toprule
		& k=2    & k=3    & k=4    \\ \midrule
		Lloyd's & 0.7762 & 0.8120 & 0.6509 \\
		EM      & 0.8260 & 0.6933 & 0.6081
	\end{tabular}
\end{table}

\section*{Section 2.3 - Differences betwene measures}
F1 is an external/supervised measure and is thus only applicable when labels are available, whereas the silhouette coefficients can always be used.

\section*{Section 3 - Compression}
The image was compressed from 127983 bytes to 72115 bytes which gived a compression ration of 1.775.

This was done py clustering all pixels in the image and then representing each pixel with the centroid it belongs to.

\section*{Section 4 - generating images from EM}
In the EM algorithm the clusters represents probability distributions. When clustering the MNIST dataset, the clusters generated seem to represent the data well, as digit like figures are generated when sampling from the probability distributions.
\end{document}